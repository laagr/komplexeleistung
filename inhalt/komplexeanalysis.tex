\section{Komplexe Zahlen und $e^{ix}$}
Eulers Formel ist ein grundlegendes Konzept in der komplexen Analyse. Es verknüpft die trigonometrischen Funktionen von Sinus und Cosinus mit der komplexen Exponentialfunktion. Es gilt, dass $e^{i\theta} = i\sin\theta + \cos\theta$ ist.
Diese Formel wird zum Beispiel verwendet, um die Nullstellen von Polynomgleichungen mit Sinus- und Cosinusfunktionen zu berechnen. Sie kann auch verwendet werden, um die Fourier-Transformation einer periodischen Funktion zu erschließen, welches wichtig für das Arbeiten mit Schallwellen und elektromagnetischen Wellen ist. Darüber hinaus bietet die Formel von Euler eine bequeme Möglichkeit, komplexe exponentielle Ausdrücke in einfachere Ausdrücke zu verwandeln.
\subsection{Die Funktion $e^{ix}$}
Was sind eigentlich komplexe Exponenten? Sicherlich multipliziert man $e$ nicht $\sqrt{-1}\pi$ mal mit sich selbst? Hierfür wird die Summation aus dem voherigen Abschnitt genommen, um die Funktion in den komplexen Raum zu erweitern. \[
  e^{ix} = \sum_{n=0}^\infty \frac{(ix)^n}{n!}
  \] Wer mit den Summationen vertraut ist, die $\cos$ und  $\sin$ definieren, der kann sich die Gleichheit von $e^{ix}$ und  $i\sin{x}+ \cos{x}$ mit dieser Formel herleiten.  \[
\sum_{n=0}^\infty \frac{(ix)^n}{n!} = (\frac{ix}{1!} - \frac{ix^3}{3!} + \frac{ix^5}{5!} + \dots) + (1 - \frac{x^2}{2!} + \frac{x^4}{4!} - \frac{x^6}{6!} + \dots) = i\sin{x} + \cos{x}
\] Eine intuitivere Erklärung erhält man durch das Ableiten von $e^{ix}$. Da  $\frac{d}{dx}e^{ix} = ie^{ix}$ ist und das Multiplizieren mit $i$ im komplexen Raum die Rotation von 90° symbolisiert, besitzt diese Funktion die Eigenschaft, dass Tangenten an der Funktion stets 90° oder senkrecht 
zu einer gegebenen Position anliegen. Zusammen mit dem Startpunkt (0, 1) ergibt das nur eine Möglichkeit: den Einheitskreis im komplexen Raum.
