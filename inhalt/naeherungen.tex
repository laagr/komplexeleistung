\section{Die Eulersche Zahl auf 1.000.000 Stellen}
Bevor ich die Eulersche Zahl auf eine Million stellen berechnen wollte, hatte ich sie mithilfe eines Python-moduls für arbiträre Präzision bereits auf einhunderttausend Stellen berechnet. Durch einen Verwandten dann über den sog. Tröpfelalgorithmus hingewiesen, mit dem man die Eulersche Zahl von allein in arbiträrer Präzision berechnen kann.
\subsection{Der Tröpfelalgorithmus}
Der Tröpfelalgorithmus funktioniert folgendermaßen:
\begin{table}[h]
\centering
\begin{tabular}{|l|llllll|}
\hline
\rowcolor[HTML]{CBCEFB} 
  & 2 & 3 & 4 & 5 & 6 & 7 \\ \hline
2 & 1 & 1 & 1 & 1 & 1 & 1 \\
\rowcolor[HTML]{DAE8FC} 
7 & 0 & 1 & 0 & 1 & 5 & 3 \\
1 & 1 & 1 & 3 & 4 & 0 & 1 \\
\rowcolor[HTML]{DAE8FC} 
8 & 0 & 1 & 2 & 0 & 2 & 6 \\ \hline
\end{tabular}
\caption{Kleine Tabelle zur Veranschaulichung / als Beispiel}
\end{table}
\par Erstelle eine Tabelle. Reihe 0 besteht aus einer leeren Stelle am Anfang und danach 2,3,4,... bis zu einer Zahl n, sodass n! die so viele Stellen hat wie man Nachkommastellen berechnen möchte. Reihe 1 beginnt mit einer 2 (welches bereits die Vorkommastelle von $e$ darstellt) und ist von nur 1 gefolgt. Um die Nächste reihe Zu berechnen beginne ganz Rechts in der jetzigen Spalte und multipliziere das Feld mit 10. Teile das durch das Feld darüber in Zeile 0, schreibe den Rest in das Untere Feld und addiere den Übertrag zum Feld rechts, Nachdem du den wert dort ebenfalls mal 10 multipliziert hat (im falle man ist ganz Rechts angekommen, dann schreibt man den übertrag unten rechts in die Reihe der Nachkommastellen). wenn man Diesen Vorgang wiederholt, dann erscheinen die Ziffern von $e$ in basis 10 als die Erste Reihe der Tabelle.

