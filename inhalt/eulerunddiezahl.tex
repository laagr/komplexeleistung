\section{Euler und die Eulersche Zahl}
Der Mathematiker, der die größte Ansammlung an mathematischen Werken und Publikationen geschrieben hat, ist Leonard Euler - Ein Schweizer Mathematiker, der einen Großteil seines Lebens in St. Petersburg arbeitete. Zu seinen Errungenschaften zählen unter Anderem: 
\subsection{Der Beweis, dass $e$ irrational ist}
Euler bewies dies, indem er zeigte, dass der einfache Kettenbruch von $e$ nicht endlich ist, denn wäre der Kettenbruch endlich, dann könnte man ihn zu einem Bruch $\frac{a}{b}$ mit $a,b \in \mathbb{Z}$ vereinfachen - also nicht rational. Dass $e$ transzendent ist (das heißt, $e$ ist nicht die Nullstelle eines Polynoms mit rationalen Koeffizienten), konnte Euler jedoch nicht beweisen. \[
e = 2 +\frac{1}{1 + \frac{1}{2 + \frac{1}{1 + \frac{1}{1 + \frac{1}{4 + \frac{1}{\ddots}}}}}} \] 
oder in vereinfachter Schreibweise: \[
e = [2; 1,2,1,1,4,1,1,\dots,\overline{2n,1,1}] \]
\subsection{Euler's Formel und Identität}
\[ e^{i\pi} + 1 = 0 \]
Die Identität besagt in einem visuellen Sinn, dass das Ergebnis von 1 aus $\pi$ Radianten um den Einheitskreis läuft, der im Feld der komplexen Zahlen liegt, also zu -1, und dann einen Schritt nach vorn nimmt und 0 wird.
\begin{figure}[h]
\includegraphics{medien/eulersidentität.pdf}
\centering
\caption{Eulers Identität}
\end{figure}
\newpage
\par Die Formel besagt, dass $e$ hoch $i\theta$ als Ergebnis die Zahl Theta Radianten um den Einheitskreis im Feld der komplexen Zahlen wiedergibt (ein Radiant ist die "` Einheit "' des Bogemaßes - 2$\pi$ Radianten sind also eine Umdrehung oder  360° um den Kreis).
\[ e^{i\theta} = i\sin{\theta} + \cos{\theta} \] 
\subsection{Das Lösen des Basler Problems}
 \[ \lim_{x\to\infty} 1 + \frac{1}{2^2} + \frac{1}{3^2} + \dots + \frac{1}{x^2} = \sum_{n=1}^{\infty} \frac{1}{n^2} = \frac{\pi^2}{6}\]
Das Basler Problem handelt von der Frage, was die Summe aller Natürlichen Zahlen ist, wenn man ihr Reziproke in's Quadrat nimmt. Euler bewies diese Summe sei $\frac{\pi^2}{6}$.
\subsection{Die Zeta Funktion}
    \[\zeta(s) = \sum_{n=1}^{\infty} \frac{1}{n^s} = 1 + \frac{1}{n} + \frac{1}{n^2} + \frac{1}{n^3} + \dots \]
Die Zeta-Funktion ist weitaus komplexer, als diese einfache Beschreibung vermuten lässt. Sie hat interessante Eigenschaften. Es wird z.B vermutet, dass die Nullstellen der Zeta Funktion unter der Eingabe von Komplexen Zahlen alle den reellen Part $\frac{1}{2}$ besitzen. Sie ist in der Zahlentheorie, der Analysis und der theoretischen Physik von großer Bedeutung. Sie ist eng mit der Verteilung der Primzahlen, der Riemannschen Vermutung und der Theorie der harmonischen Reihe verknüpft.
\subsection{Die Gamma Funktion}
\[ \Gamma(z) = \int_0^\infty e^{-t} t^{z-1} dt \] 
Die Gamma-Funktion findet Anwendung in vielen Bereichen der Mathematik, Physik und Ingenieurwissenschaften. Sie ist eine Verallgemeinerung der Fakultätsfunktion ($n!$), doch während die Fakultätsfunktion nur für positive ganze Zahlen definiert ist, kann die Gamma-Funktion für eine breitere Klasse von reellen und komplexen Zahlen verwendet werden.
\subsection{Weitere Erkenntisse von Euler}
\par Darüber hinaus leistete Euler auch bedeutende Beiträge zur Physik und Astronomie. Nennenswert sind zum Beispiel seine Arbeiten über Fluiddynamik, Optik und das Dreikörperproblem. Die Eulersche Zahl selbst dagegen wurde zum ersten Mal in einer logarithmischen Tabelle von John Napier im Jahr 1618 erwähnt. Die Konstante selbst wurde aber von Jakob Bernoulli mit der folgenden Formel eingeführt. \[
e = \lim_{n\to\infty}(1+\frac{1}{n})^n \]
