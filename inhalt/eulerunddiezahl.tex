\section{Euler und die Eulersche Zahl}
Der Mathematiker, der die größte Ansammlung an Mathematischen Werken und Publikationen geschrieben hat, ist Leonard Euler - eine Person, die seit 240 Jahren tot ist. Zu seinen Errungenschaften zählen unter Anderem: 
\begin{itemize}
  \item der Beweis, dass $e$ irrational ist:
\end{itemize}
Euler bewies dies, indem er zeigte, dass der einfache Kettenbruch von $e$ nicht periodisch und nicht endlich ist, denn wäre der Kettenbruch endlich, dann könnte man ihn zu einem Bruch $\frac{a}{b}$ mit $a,b \in \mathbb{Z}$ vereinfachen. Dass $e$ transzendent ist, konnte Euler jedoch nicht beweisen. \[
e = 2 +\frac{1}{1 + \frac{1}{2 + \frac{1}{1 + \frac{1}{1 + \frac{1}{4 + \frac{1}{\ddots}}}}}} \] 
oder in vereinfachter Schreibweise: \[
e = [2,1,2,1,1,4,1,1,\dots,\overline{2n,1,1}] \]
\begin{itemize}
  \item das Aufstellen von Euler's Formel und Identität:
\end{itemize}\[
e^{i\pi} + 1 = 0 \]
Die Identität besagt in einem visuellen Sinn, dass das Ergebnis von 1 aus $\pi$ Radianten um den Einheitskreis läuft der im Feld der komplexen Zahlen liegt, also zu -1, und dann einen Schritt nach vorne nimmt und 0 wird.
\begin{figure}[h]
\includegraphics{medien/eulersidentität.pdf}
\centering
\caption{Eulers Identität}
\end{figure}
\newpage
\par Die Formel besagt, dass $e$ hoch $i\theta$ als Ergebnis die Zahl Theta Radianten um den Einheitskreis im Feld der komplexen Zahlen wiedergibt.
\[ e^{i\theta} = i\sin{\theta} + \cos{\theta} \] 
\begin{itemize}
  \item das Lösen des Basler Problems:
\end{itemize} \[
\lim_{x\to\infty} 1 + \frac{1}{2^2} + \frac{1}{3^2} + \dots + \frac{1}{x^2} = \sum_{n=1}^{\infty} \frac{1}{n^2} = \frac{\pi^2}{6}\]
Das Basler Problem handelt mit die Frage, was die Summe aller Natürlichen Zahlen ist, wenn man ihr Reziproge in's Quadrat nimmt. Euler bewies diese Summe sei $\frac{\pi^2}{6}$.
\begin{itemize}
  \item die mit dem Basler Problem in Verbindung stehende Zeta Funktion:
\end{itemize} \[
\zeta(s) = \sum_{n=1}^{\infty} \frac{1}{n^s} = 1 + \frac{1}{n} + \frac{1}{n^2} + \frac{1}{n^3} + \dots \]
Die Zeta-Funktion ist weitaus komplexer, als diese einfache Beschreibung vermuten lässt. Sie hat interessante Eigenschaften und ist in der Zahlentheorie, der Analysis und der theoretischen Physik von großer Bedeutung. Sie ist eng mit der Verteilung der Primzahlen, der Riemannschen Vermutung und der Theorie der harmonischen Reihe verknüpft.
\begin{itemize}
  \item und die Gamma Funktion:
\end{itemize} \[
\Gamma(z) = \int_0^\infty e^{-t} t^{z-1} dt \] 
Die Gamma-Funktion findet Anwendung in vielen Bereichen der Mathematik, Physik und Ingenieurwissenschaften. Sie ist eine Verallgemeinerung der Fakultätsfunktion ($n!$), doch während die Fakultätsfunktion nur für positive ganze Zahlen definiert ist, kann die Gamma-Funktion für eine breitere Klasse von reellen und komplexen Zahlen verwendet werden.
\par Darüber hinaus leistete Euler auch bedeutende Beiträge zur Physik und Astronomie. Nennenswert sind zum Beispiel seine Arbeiten über Fluiddynamik, Optik und das Dreikörperproblem. Die Eulersche Zahl selbst dagegen wurde zum ersten Mal in einer logarithmischen Tabelle von John Napier im Jahr 1618 erwähnt. Die Konstante selbst wurde aber von Jakob Bernoulli mit der folgenden Formel eingeführt. \[
e = \lim_{x\to\infty}(1+\frac{1}{n})^n \]
